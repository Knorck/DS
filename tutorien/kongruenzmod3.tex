\documentclass[a4paper,graphics,11pt]{article}

\usepackage[utf8]{inputenc}
\usepackage[T1]{fontenc}
\usepackage{lmodern}
\usepackage[ngerman]{babel}
\usepackage{amsmath, tabu}
\usepackage{amsthm}
\usepackage{amssymb}
\usepackage{algorithm}
\usepackage{algpseudocode}
\usepackage{mathtools}
\usepackage{setspace}
\usepackage{graphicx,color,curves,epsf,float,rotating}

\floatname{algorithm}{Algorithmus}

\newcommand\norm[1]{\left\lVert#1\right\rVert}
\newcommand\abs[1]{\left\vert#1\right\vert}

\newcommand\aufgabe[1]{\subsection*{Aufgabe #1}}
\newcommand\aufgabenteil[1]{\subsubsection*{#1}}



\pagestyle{empty}
\begin{document}
\noindent WS 2016/17        							\hfill DS Tutorium \\
\noindent Woche \fbox{\textbf{5}}                      

\begin{center}
Kongruenz modulo 3
\end{center}
\aufgabenteil{a}
Zu Beweisen: $\equiv_{3}$ ist {\"A}quivalenzrelation.\\
$x,y \in \mathbb{Z}, x \equiv_{3} y$ genau dann, wenn ein $q \in \mathbb{Z}$ mit $x=q \cdot 3+y$ existiert.\\
Seien $x,y,z \in \mathbb{Z}$ mit $x \equiv_{3} y$ und $y \equiv_{3} z$, das hei{\ss}t, es existieren $p,q \in \mathbb{Z}$, sodass $x=p \cdot 3+y$ und $y=q \cdot 3+z$ gilt.\\
\fbox{\textbf{T}} 
$x=p \cdot 3+y = p \cdot 3+q \cdot 3+z = (\underbrace{p+q}_{r}) \cdot 3+z$, also $x \equiv_{3} z$, ergo ist $\equiv_{3}$ transitiv.\\
\fbox{\textbf{R}}
F{\"u}r alle $x \in \mathbb{Z}$ gilt $x=0 \cdot 3+x=x$, also $x \equiv_{3} x$, ergo ist $\equiv_{3}$ reflexiv.\\
\fbox{\textbf{S}}
Sei $x \equiv_{3} y$, also $x=q \cdot 3+y$.
$y=-q \cdot 3+x=(-q) \cdot 3+x$, also $y \equiv_{3} x$, also ist $\equiv_{3}$ symmetrisch.
\aufgabenteil{b}
$[1]=[-2]$ in $\mathbb{Z} \setminus \equiv_{3}$.
Wegen $1=1 \cdot 3 + (-2) = 1 \equiv_{3} -2$, also $[1]=[-2]$ in $\mathbb{Z} \setminus \equiv_{3}$ nach (Prop 5.6 b)).
\end{document}
