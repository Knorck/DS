\documentclass[a4paper,graphics,11pt]{article}

\usepackage[latin1]{inputenc}
\usepackage[T1]{fontenc}
\usepackage{lmodern}
\usepackage[ngerman]{babel}
\usepackage{amsmath, tabu}
\usepackage{amsthm}
\usepackage{amssymb}
\usepackage{algorithm}
\usepackage{algpseudocode}
\usepackage{mathtools}
\usepackage{setspace}
\usepackage{graphicx,color,curves,epsf,float,rotating}
\usepackage{fullpage}

\floatname{algorithm}{Algorithmus}

\newcommand\norm[1]{\left\lVert#1\right\rVert}
\newcommand\abs[1]{\left\vert#1\right\vert}

\newcommand\aufgabe[1]{\subsection*{Aufgabe #1}}
\newcommand\aufgabenteil[1]{\subsubsection*{#1}}
\newcommand{\dotcup}{\ensuremath{\mathaccent\cdot\cup}}


\pagestyle{empty}
\begin{document}
\begin{center}Tutorium Woche \fbox{12}\end{center}

\aufgabe{60}
Es sei ein K�rper $K$ gegeben. Ferner seien $A \in K^{4 \times 4}, b \in K^{4 \times 1}$ gegeben durch \begin{center}
$A =\begin{pmatrix}
	-1 & 0 & -5 & -9 \\
	-4 & -2 & 4 & -10 \\
	-4 & -4 & 4 & -11 \\
	2 & 4 & 4 & 12
\end{pmatrix}$,
$b =\begin{pmatrix}
	5 \\ 6 \\ 4 \\ -1
\end{pmatrix}$. \end{center}
Bestimmen Sie $Sol(A,b)$.\\
\emph{Hinweis}. Machen Sie Fallunterscheidungen von der Form "Wenn 2=0 in $K$ ist, dann gilt [...]".\\

$\begin{pmatrix}
	-1 & 0 & -5 & -9 | 5\\
	-4 & -2 & 4 & -10 | 6 \\
	-4 & -4 & 4 & -11 | 4\\
	2 & 4 & 4 & 12 | -4
\end{pmatrix}
add_{2,1,-4}, add_{3,1,-4}, add_{4,1,2}$\\
$= \begin{pmatrix}
	-1 & 0 & -5 & -9 & 5 \\
	0 & -2 & 24 & 26 & -14 \\
	0 & -4 & 24 & 25 & -16 \\
	0 & 4 & -6 & -6 & 6
\end{pmatrix}
add_{4,2,2},add_{3,2,-2}$\\
$= \begin{pmatrix}
	-1 & 0 & -5 & -9 |5 \\
	0 & -2 & 24 & 26 | -14 \\
	0 & 0 & -24 & -27 | 12 \\
	0 & 0 & 42 & 46 | -22
\end{pmatrix}
add_{4,3,2}
= \begin{pmatrix}
1 & 0 & -5 & -9 | 5 \\
0 & -2 & 24 & 25 | -14 \\
0 & 0 & -24 & -27 | 12 \\
0 & 0 & -6 & -8 | 2 
\end{pmatrix}
add_{3,4,-4},sw_{3,4},mult_{1,-1},mult{2,-1},\\mult_{3,-1}
= \begin{pmatrix}
1 & 0 & 5 & 9 | -5 \\
0 & 2 & -24 & -26 | 14 \\
0 & 0 & 6 & 8 | -2 \\
0 & 0 & 0 & 5 | 4 
\end{pmatrix}.$\\ \\
Sei 2=0 in $K$.\\ \\
$\begin{pmatrix} 1 & 0 & 1 & 1 & -1\\0 & 0 & 0 & 1 & 0\\0 & 0 & 0 & 0 & 0\\0 & 0 & 0 & 1 & 0 \end{pmatrix}
add_{4,2,1}... \mapsto \begin{pmatrix} 1 & 0 & 1 & 1 & -1\\0 & 0 & 0 & 1 & 0\\0 & 0 & 0 & 0 & 0\\0 & 0 & 0 & 0 & 0 \end{pmatrix}$.\\
$Sol(A,b)=\{\begin{pmatrix} 1+b \\ a \\ b \\ 0\end{pmatrix}| a,b \in K\}=\begin{pmatrix} 1 \\ 0 \\ 0 \\ 0 \end{pmatrix} + a \cdot \begin{pmatrix} 0 \\ 1 \\ 0 \\ 0 \end{pmatrix} + k \begin{pmatrix} -1 \\ 0 \\ 1 \\ 0 \end{pmatrix}$. \\ \\
Sei 3=0 in $K$. \\ \\
$\begin{pmatrix} 1 & 0 & -1 & 1 & 0 \\ 0 & -1 & 0 & 1 & -1 \\ 0 & 0 & 0 & 1 & -1 \\ 0 & 0 & 0 & -1 & 1 \end{pmatrix}
add_{1,3,-1},add_{2,3,-1},add_{4,3,1},mult_{2,-1} \mapsto
\begin{pmatrix} 1 & 0 & -1 & 0 & 1 \\ 0 & -1 & 0 & 0 & 0 \\ 0 & 0 & 0 & 1 & -1 \\ 0 & 0 & 0 & 0 & 0 \end{pmatrix}$.\\
$Sol(A,b)=\{\begin{pmatrix} 1+a \\ 0 \\ a \\ -1 \end{pmatrix}| a \in K\}=
\begin{pmatrix} 1 \\ 0 \\ 0 \\ -1 \end{pmatrix} + k \cdot \begin{pmatrix} 1 \\ 0 \\ 1 \\ 0 \end{pmatrix}$. \\ \\
Sei 5 = 0 in $K$. \\ \\
$\begin{pmatrix} 1 & 0 & -1 & 1 & 2 \\ 0 & 2 & 0 & 1 & 2 \\ 0 & 0 & 1 & -2 & 0 \\ 0 & 0 & 0 & 0 & -1 \end{pmatrix}$.\\
F�r $x\in Sol(A,b)$ gilt -1=0 in $K$. Widerspruch.�\\
$Sol(A,b)=\{\}$.\\ \\
Sei $2 \neq 0, 3 \neq 0, 5�\neq 0$ in $K$. \\ \\
$\begin{pmatrix} 1 & 0 & -1 & 1 & -3 \\ 0 & 2 & 0 & 1 & 2 \\ 0 & 0 & 6 & -2 & -10 \\ 0 & 0 & 0 & 5 & 4 \end{pmatrix}
mult_{4,5^{-1}},add_{1,4,-1},add_{2,4,-1^{0}},add_{3,4,-1^{0}}...\mapsto
\begin{pmatrix} 1 & 0 & 0 & 0 & -5^{-1} \cdot 26 \\ 0 & 1 & 0 & 0 & 5^{-1} \cdot 3 \\
0 & 0 & 1 & 0 & -5^{-1} \cdot 7 \\ 0 & 0 & 0 & 1 & 5^{-1} \cdot 4 \end{pmatrix}$.\\ \\ 
$Sol(A,b)=\{5^{-1} \cdot \begin{pmatrix} -26 \\ 3 \\ -7 \\ 4 \end{pmatrix}\}$.
\aufgabe{61}
Es seien ein K�rper $K$ und $A \in K^{2 \times 2}$ gegeben. Zeigen Sie, dass folgende Bedingungen �quivalent sind.
\begin{itemize}
\item (a) F�r alle $b \in K^{2 \times 1}$ ist gibt es genau eine L�sung des linearen Gleichungssystems zur erweiterten Koeffizientenmatrix $(A|b)$.
\item (b) Die L�sungsmenge des homogenen linearen Gleichungssystems zur Koeffizientenmatrix $A$ ist gegeben durch
\begin{center}$Sol(A,b)=\{0\}$. \end{center}
\item (c) Es gilt \begin{center} $A_{1,1}A_{2,2}-A_{1,2}A_{2,1} \neq 0$. \end{center}
\end{itemize}
\aufgabe{des Tutors}
\begin{itemize}
\item 4 Vorspeisen
\item 6 Hauptgerichte
\item 3 Nachspeisen
\item Vorspeisen und Nachspeisen nicht notwendig verschieden
\end{itemize}
2 Personen bestellen:
\begin{itemize}
\item H�chstens 2 Vorspeisen
\item H�chstens 2 Nachspeisen
\item Entweder genau ein Hauptgericht oder 2 verschiedene Hauptgerichte
\end{itemize}
Sei $V$ die Menge der Vorspeisen, $H$ die Menge Haupgerichte, $N$ die Menge der Nachspeisen.\\
$B_{Vorspeisen}=MComb_{0}(V) \dotcup MComb_{1}(V) \dotcup MComb_{2}(V)$.\\
$B_{Nachspeisen}=MComb_{0}(N) \cup MComb_{1}(N) \dotcup MComb_{2}(V)$.\\
$B_{Hauptgerichte}=Comb_{1}(H) \dotcup Comb_{2}(H)$.\\ \\
$B=B_{Vorspeisen} \times B_{Nachspeisen} \times B_{Hauptgerichte}$.
Summenregel (16.4), Korollar (16.61), Korollar (16.54):\\
$|B_{Vorspeisen}|=|MComb_{0}(V) \dotcup MComb_{1}(V) \dotcup MComb_{2}(V)|=\\
|MComb_{0}(V)|+|MComb_{1}(V)|+MComb_{2}(V)|=\\
\binom{0+4-1}{0}+\binom{1+4-1}{1}+\binom{2+4-1}{2}=\binom{3}{0}+\binom{4}{1}+\binom{5}{2}=1+4+10=15$. \\ \\
$|B_{Nachspeisen}|=....=\binom{0+3-1}{0}=\binom{1+3-1}{1}+\binom{2+3-1}{2}= \binom{2}{0}+\binom{3}{1}+\binom{4}{2}\\
=1+3+6=10$.\\ \\
$|B_{Hauptspeise}|=|Comb_{1}(H) \dotcup Comb_{2}(H)|=|Comb_{1}(H)|=|Comb_{2}(H)|\\
=\binom{6}{1}+\binom{6}{2}= 6+15=21$.\\
$|B|=|B_{Vorspeisen} \times B_{Nachspeisen} \times B_{Hauptgerichte}|\\
=|B_{Vorspeisen}| \cdot |B_{Nachspeisen}| \cdot |B_{Hauptgerichte}|= 15 \cdot 10 \cdot 21 = 3150$.
\end{document}
% Nummer des Blattes angepasst?