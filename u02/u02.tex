\documentclass[a4paper,graphics,11pt]{article}

\usepackage[utf8]{inputenc}
\usepackage[T1]{fontenc}
\usepackage{lmodern}
\usepackage[ngerman]{babel}
\usepackage{amsmath, tabu}
\usepackage{amsthm}
\usepackage{amssymb}
\usepackage{algorithm}
\usepackage{algpseudocode}
\usepackage{mathtools}
\usepackage{setspace}
\usepackage{graphicx,color,curves,epsf,float,rotating}

\floatname{algorithm}{Algorithmus}

\newcommand\norm[1]{\left\lVert#1\right\rVert}
\newcommand\abs[1]{\left\vert#1\right\vert}

\newcommand\aufgabe[1]{\subsection*{Aufgabe #1}}
\newcommand\aufgabenteil[1]{\subsubsection*{#1}}



\pagestyle{empty}
\begin{document}
\noindent WS 2016/17        							\hfill Simon Kaiser, 354692 \\
\noindent Gruppe 3                                      \hfill Johannes Salentin, 367543 \\

\begin{center}
\Large \textsc{Diskrete Strukturen} \\   % Fach
\large Aufgabenblatt 2                        % Nummer das Blattes, nicht vergessen zu �ndern!
\end{center}
\begin{center}
\rule[0.5ex]{\textwidth}{0.6pt}\vspace*{-\baselineskip}\vspace{3.2pt}
\rule[0.5ex]{\textwidth}{1.6pt}\\
\end{center}


%%%%%%%%%%%%%%%%%%%%%%%%%%%%%%%%%%%%%%
%
%   Ab hier kommt der Text
%   Neue Aufgabe mit \aufgabe{}
%   Aufgabenteil mit \aufgabenteil{}
% 
%%%%%%%%%%%%%%%%%%%%%%%%%%%%%%%%%%%%%%

\aufgabe{11}

Bestimmen Sie eine aussagenlogische Formel in dijkuntiver Normalform und eine aussagenlogische Formel in konjunktiver Normalform, welche jeweils logisch {\"a}quivalent zu $\neg(\neg B \lor A \iff \neg C)$ sind. \\

\begin{tabular}[c]{l l l || c | c | c } 
$ A & B & C & \neg B\lor A & \neg B \lor A \iff \neg C & \neg ( \neg B \lor A \iff \neg C ) $\\ 
\hline
 1 & 1 & 1 & 1 & 0 & 1 \\
1 & 1 & 0 & 1 & 1 & 0 \\
1 & 0 & 1 & 1 & 0 & 1 \\
1 & 0 & 0 & 1 & 1 & 0 \\
0 & 1 & 1 & 0 & 1 & 0 \\
0 & 1 & 0 & 0 & 0 & 1 \\
0 & 0 & 1 & 1 & 0 & 1 \\
0 & 0 & 0 & 1 & 1 & 0 \\
\end{tabular}
\\

Aus den Wahrheitswerten der aussagenlogischen Formel \\
$ \neg ( \neg B \lor A \iff \neg C ) $
folgt: \\
Konjunkive Normalform: 
$Con(110) \land Con(100) \land Con(011) \land Con(000) \\ $
$ \iff (\neg A \lor \neg B \lor C) \land ( \neg A \lor B \lor C) \land (A \lor \neg B \lor \neg C) \land (A \lor B \lor C). $

\aufgabe{12}

\aufgabenteil{a}
Seien A, B, C, D, X Mengen, wobei $ A, B, C, D \subseteq X$ gilt. \\
Die Behauptung ist $(A \times C) \cup (B \times D) = (A \cup B) \times (C \cup D)$. \\
Seien die Mengen $A := \{1,2\}, B:= \{3,4\}, C := \{5,6\}$ und $D := \{7,8\}$ gegeben. \\
Angenommen: 
$$\{(1,5),(1,6),(2,5),(2,6)\} \cup \{(3,7),(3,8),(4,7),(4,8)\} = \{1,2,3,4\} \times \{5,6,7,8\} $$
$$\iff\{(1,5),(1,6),(2,5),(2,6),(3,7),(3,8),(4,7),(4,8)\} \\ $$ 

\begin{center} $ = \{(1,5),(1,6),$\underbrace{(1,7)}_{nicht~in~linker~Menge~enthalten}},...\}. $ \\
\end{center}
Daraus folgt, dass die Mengen verschieden sind. Somit ist die Behauptung widerlegt.
\aufgabenteil{b}
A, B, C seien beliebige Mengen.\\
Die Behauptung ist $ A \cup (B \cap C) = (A \cup B) \cap (A \cup C)$. 
$$ A \cup (B \cap C) = A \cup \{x |(x \in B \land x \in C\} $$

\begin{aligned}
$$ A \cup \{x |(x \in B \land x \in C\} &= \{x|x \in A \lor (x \in B \land x \in C)} $$ \\
$$&= \{x |(x \in A \lor x \in B) \land (x \in A \lor x \in C)\} $$ \\
$$&= \{x |(x \in A \cup B) \land (x \in A \cup C)\} $$ \\
$$&= (A \cup B) \cap (A \cup C) ~~~\Box $$ \\
\end{aligned}

\aufgabenteil{c}
A, B, C seien beliebige Mengen. \\
Die Behauptung ist $A \times (B \cap C) = (A \times B) \cap (A \times C)$.

\begin{aligned}
$$A \times (B \cap C) &= \{(x,y) | x \in A \land y \in (B \land C)\} $$ \\
$$&= \{(x,y) | x \in A \land y \in B \land x \in A \land y \in C\} $$ \\
$$&= \{(x,y) | x \in A \land y \in B\} \cap {(x,y) | x \in A \land y \in C\} $$ \\
$$&= (A \times B) \cap (A \times C) ~~~\Box $$ \\
\end{aligned}

\aufgabenteil{d}
Seien A, B, X Mengen, wobei $ A, B \subseteq X$ gilt. \\
Die Behauptung ist $X \setminus (A \cup B) = (X \setminus A) \cap (X \setminus B)$.
$$ &X \setminus (A \cup B) = \{x | x \in X \land x \notin (A \cup B)\} $$ 
Es gilt \(\neg(A \lor B) \iff \neg A \land \neg B\)  gem{\"a}{\ss} (1.21 DeMorgan), sodass folgt: \\
\begin{aligned}
$$&=\{x | (x \in X \land x \notin A) \land (x \in X \land x \notin B)\} $$ \\
$$&=(x \setminus A) \cap (x \setminus B)~~~\Box $$ 
\end{aligned}

\aufgabenteil{e}
Die Behauptung ist $A \cup (B \setminus C) = (A \cup B) \setminus (A \cup C)$. \\
Seien die Mengen $A := \{1,2\}, B:= \{3,4\}$ und $C := \{5,6\}$ gegeben. 

\begin{aligned}
$$\{1,2\} \cup \{3,4\} &= \{1,2,3,4\} \setminus \{1,2,5,6\} $$ \\
$\{1,2,3,4\} &= \{3,4\} \rightarrow $ Widerspruch
\end{aligned}

\aufgabenteil{f}
A, B, C seien beliebige Mengen. \\
Die Behauptung ist $(A \cup B) \setminus C = (A \setminus C) \cup (B \setminus C)$.\\
$$(A \cap B) \setminus C = {x | x \in A \lor x \in B) \land x \notin C \}$$ \\
Es gilt $(A \lor B) \land C \iff (A \land C) \lor (B \land C)$ gem{\"a}{\ss} (1.18,f,(i) Distr.), sodass folgt:

\begin{aligned}
$$&=\{x | x \in A \land x \notin C \lor x \in B \land x \notin C \} $$\\
$$&=(A \setminus C) \cup (B \setminus C)~~~\Box
\end{aligned}

\end{document}

% Nummer des Blattes angepasst?