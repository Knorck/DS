\documentclass[a4paper,graphics,11pt]{article}

\usepackage[latin1]{inputenc}
\usepackage[T1]{fontenc}
\usepackage{lmodern}
\usepackage[ngerman]{babel}
\usepackage{amsmath, tabu}
\usepackage{amsthm}
\usepackage{amssymb}
\usepackage{algorithm}
\usepackage{algpseudocode}
\usepackage{mathtools}
\usepackage{setspace}
\usepackage{graphicx,color,curves,epsf,float,rotating}
\usepackage{fullpage}

\floatname{algorithm}{Algorithmus}

\newcommand\norm[1]{\left\lVert#1\right\rVert}
\newcommand\abs[1]{\left\vert#1\right\vert}

\newcommand\aufgabe[1]{\subsection*{Aufgabe #1}}
\newcommand\aufgabenteil[1]{\subsubsection*{#1}}



\pagestyle{empty}
\begin{document}
\noindent WS 2016/17        							\hfill Simon Kaiser, 354692 \\
\noindent Gruppe \fbox{\textbf{3}}                      \hfill Johannes Salentin, 367543 \\

\begin{center}
\Large \textsc{Diskrete Strukturen} \\   % Fach
\large Aufgabenblatt 6                        % Nummer das Blattes, nicht vergessen zu �ndern!
\end{center}
\begin{center}
\rule[0.5ex]{\textwidth}{0.6pt}\vspace*{-\baselineskip}\vspace{3.2pt}
\rule[0.5ex]{\textwidth}{1.6pt}\\
\end{center}


%%%%%%%%%%%%%%%%%%%%%%%%%%%%%%%%%%%%%%
%
%   Ab hier kommt der Text
%   Neue Aufgabe mit \aufgabe{}
%   Aufgabenteil mit \aufgabenteil{}
% 
%%%%%%%%%%%%%%%%%%%%%%%%%%%%%%%%%%%%%%
\aufgabe{36}
\aufgabenteil{a}
Es seien ein kommutativer Ring $R,a\in R^{\mathbb{N}},c \in R^{\mathbb{N}_{0}}$ und $x \in R^{\mathbb{N}_{0}}$ mit\\
\[ x_{k}=
\begin{cases}
c_{0}, & \text{f�r }k=0,\\
a_{k}x_{k-1}+c_{k}, & \text{f�r }k \in \mathbb{N}\\
\end{cases}
\]
gegeben. Bestimmen Sie eine Formel f�r $x$.\\
Definiert man nun $\sum_{0}^{-1}:=0$, so lautet eine geschlossene Formel f�r $x$:\\
$\displaystyle\sum_{i=0}^{k-1}[(\displaystyle\prod_{n=l+1}^{k}a_{n})\cdot c_{l}] +c_{k}$. \\
Beweis durch vollst�ndige Induktion nach $k \in \mathbb{N}_{0}$. $x_{k}=a_{k}x_{k-1}+c{k}$.\\
\emph{I.A.}\\
$k=0: \displaystyle\sum_{i=0}^{0-1}[(\displaystyle\prod_{n=l+1}^{0}a_{n})\cdot c_{l}] +c_{0}= c_{0}. \checkmark$\\
$k=1: \displaystyle\sum_{i=0}^{1-1}[(\displaystyle\prod_{n=l+1}^{1}a_{n})\cdot c_{l}] +c_{1}= a_{1} \cdot c_{0}+c_{1}. \checkmark$\\
\emph{I.S.}\\
F�r $k�\in \mathbb{N}_{0}$ mit $k \geq 2$ und $x_{k-1}=\displaystyle\sum_{i=0}^{k-1}[(\displaystyle\prod_{n=l+1}^{k-1}a_{n})\cdot c_{l}] +c_{k-1}$ gilt:\\
$x_{k}=a_{k}x{k-1}+c_{k}=a_{k}(\displaystyle\sum_{l=0}^{k-2}[(\displaystyle\prod_{n=l-1}^{k-1}a_{n}) \cdot c_{l}]+c_{k-1})+c_{k}$\\
$=\displaystyle\sum_{l=0}^{k-2}[(\displaystyle\prod_{n=l+1}^{k-1}a_{n}) \cdot c_{l} \cdot a_{k}]+c_{k-1} \cdot a_{k} + c_{k}$\\
$=\displaystyle\sum_{l=0}^{k-2}[(\displaystyle\prod_{n=l+1}^{k-1}a_{n}) \cdot c_{l}]+c_{k}$\\
$=\displaystyle\sum_{l=0}^{k-1}[(\displaystyle\prod_{n=l+1}^{k-1}a_{n}) \cdot c_{l}]+c_{k}$\\
Nach dem Prinzip der vollst�ndigen Induktion ist die ausgegebene Formel eine geschlossene Formel f�r $x$.\hfill $\Box$\\
\aufgabenteil{b}
Es seien ein kommutativer Ring $R$, $a�\in R$, $c \in R^{\mathbb{N}_{0}}$, $n \in \mathbb{N}$ mit $n \geq 2$ und $x \in R^{\mathbb{N}_{0}}$ mit
\[ x_{k}=
\begin{cases}
c_{0}, & \text{f�r }k=0,\\
ax_{\lfloor k/n \rfloor}+c_{k}, & \text{f�r }k \in \mathbb{N}.
\end{cases}
\]
Ist nun also $x_{n^{l}}=ax_{n^{l-1}}+c_{n^{l}}$, so ist $x_{n^{l}}=\sum_{i=0}^{l}[(\prod_{m=i}^{l}a \cdot c_{\lfloor n^{i-1} \rfloor} ] + c_{n^{l}}$ eine geschlossene Formel f�r $(x_{n^{l}})_{l \in \mathbb{N}_{0}}$.\\
Beweis per vollst�ndiger Induktion nach $l \in \mathbb{N}_{0}$.\\ \\
\emph{I.A.}
\begin{center}
$\begin{aligned}$
$l=0:x_{n^{0}}&=\sum_{i=0}^{0}[(\prod_{m=i}^{0}a \cdot c_{\lfloor n^{i-1} \rfloor} ] + c_{n^{0}}$\\
$&= a \cdot c_{0} + c_{n^{0}}$. \checkmark \\
$\end{aligned}$
\\
$\begin{aligned}$
$l=1:x_{n^{1}}&=\sum_{i=0}^{1}[(\prod_{m=i}^{1}a \cdot c_{\lfloor n^{i-1} \rfloor} ] + c_{n^{1}}$\\
$&=a^{2} \cdot c_{0} + a \cdot c_{n^{0}}+c_{n^{1}}. \checkmark \\
$\end{aligned}$
\end{center}
\emph{I.S.}\\
F�r $l \geq 2$ und $x_{n^{l-1}}=\sum_{i=0}^{l-1}[(\prod_{m=i}^{l-1}a \cdot c_{\lfloor n^{i-1} \rfloor} ] + c_{n^{l-1}}$ gilt\\
\begin{center}
$\begin{aligned}$
$x_{n^{l}}&=ax_{n^{l-1}}+c_{n^{l}}=a{x_{n^{l-1}}=\sum_{i=0}^{l-1}[(\prod_{m=i}^{l-1}a \cdot c_{\lfloor n^{i-1} \rfloor} ] + c_{n^{l-1}})+c_{n^{l}}$\\
$&=x_{n^{l-1}}=\displaystyle\sum_{i=0}^{l-1}[(\displaystyle\prod_{m=i}^{l-1}a \cdot a \cdot c_{\lfloor n^{i-1} \rfloor} ] +a \cdot c_{n^{l-1}}+c_{n^{l}}$\\
$&=x_{n^{l}} =\sum_{i=0}^{l}[(\prod_{m=i}^{l}a \cdot c_{\lfloor n^{i-1} \rfloor} ] + c_{n^{l}}$. \\
$\end{aligned}$
\end{center}
\\
Mit dem Prinzip der vollst�ndigen Induktion ist die Aussage bewiesen. \hfill $\Box$ 
\aufgabenteil{c}
\aufgabe{37}
Es seien $f,g \in \mathbb{Q}[X]$ gegeben durch\\
$f=2X^{5}-5X^{4}-11X^{3}+4X{2}+6X-2$,\\
$g=2X^{3}-5X^{2}-9X-2$.\\ \\
Es gilt $gcd(f,g)=gcd(g,f~mod~g)$.\\
$(2X^{5}-5X^{4}-11X^{3}+4X{2}+6X-2)/(2X^{3}-5X^{2}-9X-2) = X^{2}-1$.\\
Daraus folgt $2X^{5}-5X^{4}-11X^{3}+4X{2}+6X-2=(X^{2}-1)(2X^{2}-5X^{2}-9X-2)+(X^{2}-3X-4)$.\\
$(2X^{3}-5X^{2}-9X-2)/(X^{2}-3X-4)=2X+1$.\\
Daraus folgt $2X^{3}-5X^{2}-9X-2=(2X+1)(X^{2}-3X4)+2X+2$.\\
$(X^{2}-3X-4)/(2X+2)=0.5X-2$.\\
Daraus folgt $X^{2}-3X-4=(2X+2)(0.5X-2)+0$.\\
Da der Rest der Polynomdivision gleich Null ist, ist $gcd(f,g)=2X+2$.\\ \\
$x_{0}:=1,~ x_{1}:=0,\\ x_{2}:=1, ~~x_{3}:=-2X-1.$\\
$y_{0}:=0,~~ y_{1}:=1,~~ y_{2}:=-X^{2}+1,\\y_{3}:=1-(2X+1)(-X^{2}+1)= 2X^{3}+X^{2}-2X$.\\
$r_{0}:=2X^{5}-5X^{4}-11X^{3}+4X^{2}+6X-2$,\\
$r_{1}:=2X^{3}-5X^{2}-9X-2$,\\
$r_{2}:=X^{2}-3X-4$, ~$r_{3}:=2X-2$.\\ \\
Es folgt also\\
$2X+2=r_{3}=$\\
$(-2X-1)(2X^{5}-5X^{4}-11X^{3}+4X^{2}+6X-2)+(2X^{3}+X^{2}+2X)(2X^{3}-5X^{2}-9X-2)$.\\
$\iff X+1=(-X-0.5)(2X^{5}-5X^{4}-11^{3}+4X^{2}+6X-2)+(X^{3}+0.5X^{2}-X)(2X^{3}-5X^{2}-9X-2)$.\\ \\
Also ist $h=x+1$ und $k=X^{3}+0.5X^{2}-X$, wobei $h,k \in \mathbb{Q}[X]$ und $gcd(f,g)=hf+kg$ gilt.
\end{document}
% Nummer des Blattes angepasst?