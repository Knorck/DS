\documentclass[a4paper,graphics,11pt]{article}

\usepackage[utf8]{inputenc}
\usepackage[T1]{fontenc}
\usepackage{lmodern}
\usepackage[ngerman]{babel}
\usepackage{amsmath, tabu}
\usepackage{amsthm}
\usepackage{amssymb}
\usepackage{algorithm}
\usepackage{algpseudocode}
\usepackage{mathtools}
\usepackage{setspace}
\usepackage{graphicx,color,curves,epsf,float,rotating}

\floatname{algorithm}{Algorithmus}

\newcommand\norm[1]{\left\lVert#1\right\rVert}
\newcommand\abs[1]{\left\vert#1\right\vert}

\newcommand\aufgabe[1]{\subsection*{Aufgabe #1}}
\newcommand\aufgabenteil[1]{\subsubsection*{#1}}



\pagestyle{empty}
\begin{document}
\noindent WS 2016/17        							\hfill Simon Kaiser, 354692 \\
\noindent Gruppe 3                                      \hfill Johannes Salentin, 367543 \\

\begin{center}
\Large \textsc{Diskrete Strukturen} \\   % Fach
\large Aufgabenblatt 3                        % Nummer das Blattes, nicht vergessen zu ändern!
\end{center}
\begin{center}
\rule[0.5ex]{\textwidth}{0.6pt}\vspace*{-\baselineskip}\vspace{3.2pt}
\rule[0.5ex]{\textwidth}{1.6pt}\\
\end{center}
\aufgabe{17}
\aufgabenteil{a}
Es seien \\
$f: \mathbb{Z} \to \mathbb{Z} \times \mathbb{Z}, a \mapsto (a-1,2)$ und $g: \mathbb{Z} \times \mathbb{Z} \to \mathbb{Z}, (b_{1}, b_{2}) \mapsto b_{1} + b_{2}$. \\
Untersuchen Sie, ob die Komposita $g \circ f$ und $f \circ g$ definiert sich und bestimmen Sie diese gegebenenfalls.\\
Ja, die Abbildungen sind definiert, denn die Abbildungen $f$ und $g$ sind so gegeben, dass die Zielmenge von $f$ gleich der Startmenge von $g$ ist, und die Zielmenge von $g$ gleich der Startmenge von $f$ ist. Die Komposita $g \circ f$ und $f \circ g$ sind also definiert. \\

$\begin{aligned}$
$(g \circ f)(a) &= g(f(a)) = (a-1)+2 = a+1$ f{\"u}r $a \in \mathbb{Z}$. \\
$(f \circ~g)(b_{1},b_{2}) &= f(g((b_{1},b_{2}))) = (b_{1}+b_{2}-1,2)$ f{\"u}r $(b_{1},b_{2}) \in \mathbb{Z} \times \mathbb{Z}.
$\end{aligned} $
\aufgabenteil{b}
Zeigen oder widerlegen Sie: \\
(i) \\
F{\"u}r alle Abbildungen $f,g: \{1,2,3\} \to \{1,2,3\}$ gilt $g \circ f = f \circ g$. \\
Widerlegung durch Gegenbeispiel:\\
Seien $f:\{1,2,3\} \to \{1,2,3\}, 1 \mapsto 1, 2 \mapsto 1, 3 \mapsto 1$ \\
und $g:\{1,2,3\} \to \{1,2,3\}, 1 \mapsto 2, 2 \mapsto 2, 3 \mapsto 2$ gegeben. \\
So ist $g \circ f: \{1,2,3\} \to \{1,2,3\}, 1 \mapsto 2, 2 \mapsto 2, 3 \mapsto 2$ \\
und $f \circ g:\{1,2,3\} \to \{1,2,3\}, 1 \mapsto 1, 3 \mapsto 1, 3 \mapsto 1$ nicht gleich.\\
Somit ist die Aussage widerlegt. \\
(ii)\\
F{\"u}r alle Abbildungen $f,g: \{1,2,3\} \to \{1,2,3\}$ gilt $g \circ f \neq f \circ g$.\\
Widerlegung durch Gegenbeispiel:\\
Seien $f:\{1,2,3\} \to \{1,2,3\}, 1 \mapsto 2, 2 \mapsto 3, 3 \mapsto 1$ \\
und $g:\{1,2,3\} \to \{1,2,3\}, 1 \mapsto 3, 2 \mapsto 1, 3 \mapsto 2$ gegeben. \\
So sind $g \circ f: \{1,2,3\} \to \{1,2,3\}, 1 \mapsto 1, 2 \mapsto 2, 3 \mapsto 3$\\
und $f \circ g: \{1,2,3\} \to \{1,2,3\}, 1 \mapsto 1, 2 \mapsto 2, 3 \mapsto 3$ gleich.\\
Somit ist die Aussage widerlegt.\\
Insgesamt gibt es aber Abbildungen, f{\"u}r die $g \circ f = f \circ g$ gilt, aber es gibt auch Abbildungen, f{\"u}r die genau dies nicht gilt, also $g \circ f \neq f \circ g$ ist. Folglich sind beide Aussagen widerlegt.
\aufgabenteil{c}
Zu zeigen: Die Abbildung 
$f:\{0,1\}^{\{0,1\}} \to \{0,1,2,3\}, h \mapsto h_{0} \cdot 2^0 + b_{1} \cdot 2^1$
ist invertierbar.\\ Die Abbildung ordnet jeder Abbildung mit Start- und Zielmenge $\{0,1\}$ einer Zahl von 0 bis 3 zu. F kann somit auch wie folgt geschrieben werden:\\

$\begin{aligned}
$f:\{0,1\}^{\{0,1\}} \to \{0,1,2,3\}, &(f_{0}:\{0,1\} \to \{0,1\}, 0 \mapsto 0, 1 \mapsto 0) \mapsto 0$, \\
$&(f_1:\{0,1\} \to \{0,1\}, 0 \mapsto 1, 1 \mapsto 0) \mapsto 1$,\\
$&(f_2:\{0,1\} \to \{0,1\}, 0 \mapsto 0, 1 \mapsto 1) \mapsto 2$,\\
$&(f_3:\{0,1\} \to \{0,1\}, 0 \mapsto 1, 1 \mapsto 1) \mapsto 3.\\
$\end{aligned}$

Sei nun $g$ gegeben mit:\\

$\begin{aligned}$
$g:\{0,1,2,3\} \to \{0,1\}^{\{0,1\}}, &0 \mapsto (f_{0}:\{0,1\} \to \{0,1\}, 0 \mapsto 0, 1 \mapsto 0)$,\\
$&1 \mapsto (f_{1}:\{0,1\} \to \{0,1\}, 0 \mapsto 1, 1 \mapsto 0)$,\\
$&2 \mapsto (f_{2}:\{0,1\} \to \{0,1\}, 0 \mapsto 0, 1 \mapsto 1)$,\\
$&3 \mapsto (f_{3}:\{0,1\} \to \{0,1\}, 0 \mapsto 1, 1 \mapsto 1)$.\\
$\end{aligned}$

Dann gilt $f \circ g = id_{\{0,1,2,3\}}$ und $g \circ f = id_{\{0,1\}^{\{0,1\}}}$.\\
Das bedeutet, $g$ ist eine Inverse zu $f$ und folglich ist $f$ auch invertierbar.\\
\aufgabe{18}
Sei $x \in \{\mathbb{N}, \mathbb{Z}, \mathbb{Q}}$\}.\\
$f:X \to X, x \mapsto 4x-3$.\\
Sei $X = \mathbb{N}$ oder $X = \mathbb{Z}, g: X \to X, y \mapsto (y+3) div $.\\
\begin{aligned}
$(g \circ f)(x) &= g(f(x)) = g(4x-3) = (4x-3+3)div4$\\
$&=4x~div4=x$, also $g \circ f = id_{x}$.
\end{aligned}\\
Daraus folgt, dass f injektiv ist gem{\"a}{\ss} (3.29a).\\
Es gibt kein $x \in X$ mit $4x=5$, also kein $x \in X$ mit $f(x)=4x-3=2$. Es folgt, dass f nicht surjektiv ist.\\ 
Sei nun $x=\mathbb{Q}$. $g:X\to X, y \mapsto \frac{1}{4}(y+3)$.\\
Es gilt f{\"u}r $x,y \in X: y=4x-3 \iff y+3=4x \iff \frac{y+3}{4}=x$.\\
$(f \circ g)(x) = g(f(x)) = g(4x-3) = \frac{1}{4}(4x-3+3) = \frac{1}{4}\cdot 4x = x$\\
$(f \circ g)(y) = f(g(y)) = f(\frac{1}{4}(y+3)) = y$.\\ \\
Also $g \circ f = id_{x}$ und $f \circ g = id_{x}$ nach (3.6). \\Nach (3.29c) ist $f$ also bijektiv.~~~ \Box
\end{document}
