\documentclass[a4paper,graphics,11pt]{article}

\usepackage[latin1]{inputenc}
\usepackage[T1]{fontenc}
\usepackage{lmodern}
\usepackage[ngerman]{babel}
\usepackage{amsmath, tabu}
\usepackage{amsthm}
\usepackage{amssymb}
\usepackage{algorithm}
\usepackage{algpseudocode}
\usepackage{mathtools}
\usepackage{setspace}
\usepackage{graphicx,color,curves,epsf,float,rotating}
\usepackage{fullpage}

\floatname{algorithm}{Algorithmus}

\newcommand\norm[1]{\left\lVert#1\right\rVert}
\newcommand\abs[1]{\left\vert#1\right\vert}

\newcommand\aufgabe[1]{\subsection*{Aufgabe #1}}
\newcommand\aufgabenteil[1]{\subsubsection*{#1}}



\pagestyle{empty}
\begin{document}
\noindent WS 2016/17        							\hfill Simon Kaiser, 354692 \\
\noindent Gruppe \fbox{\textbf{3}}                      \hfill Johannes Salentin, 367543 \\

\begin{center}
\Large \textsc{Diskrete Strukturen} \\   % Fach
\large Aufgabenblatt 9
\end{center}
\begin{center}
\rule[0.5ex]{\textwidth}{0.6pt}\vspace*{-\baselineskip}\vspace{3.2pt}
\rule[0.5ex]{\textwidth}{1.6pt}\\
\end{center}


%%%%%%%%%%%%%%%%%%%%%%%%%%%%%%%%%%%%%%
%
%   Ab hier kommt der Text
%   Neue Aufgabe mit \aufgabe{}
%   Aufgabenteil mit \aufgabenteil{}
% 
%%%%%%%%%%%%%%%%%%%%%%%%%%%%%%%%%%%%%%

\aufgabe{53}
Es sei $R=\mathbb{Z}$ oder $R=K[X]$ f�r einen K�rper $K$. Ferner seien $m,a,b \in R$ gegeben.
\aufgabenteil{a} Zeigen Sie, dass es genau dann ein $x \in R$ mit $xa\equiv_{m}b$ gibt, wenn $gcd(a,m)|b$ gilt.\\
Zu zeigen: Es gibt genau dann ein $x \in R$ mit $xa\equiv_{m}b$, wenn $gcd(a,m)|b$.
\begin{center}
$xa\equiv_{m}b \iff xa=b+ym \iff xa+ym=b$.
\end{center}
Nach 10.26a besitzt diese Gleichung eine L�sung, wenn $gcd(a,m)|b$ ist.
Somit ist die Aussage gezeigt. \hfill{$\Box$}
\aufgabenteil{b} Es seien $p,q,r \in R$ mit $a=p \cdot gcd(a,m)$, $m=q \cdot gcd(a,m)$, $b=r \cdot gcd(a,m)$ sowie $x'\in R$ mit $x'p\equiv_{q} 1$ gegeben. Zeigen Sie, dass
\[ \{x \in R |xa \equiv_{m}b\}=
	\begin{cases}
	R, & \text{falls }(a,m)=(0,0),\\
	rx'=Rq, & \text{falls }(a,m)\neq(0,0)
	\end{cases}
\]
ist.\\
$x' \in R$ mit $x'p \equiv_{q}1$. $xa\equiv_{m}b \iff xa-ym=b$.\\
Nach 10.26b gilt:
\[ \{(x,y) \in R \times R | xa-ym=b\}=
	\begin{cases}
	R�\times R, & \text{falls }(a,m)=(0,0),\\
	(rx'+zq, ry'-zp), & \text{falls }(a,m)\neq(0,0)
	\end{cases}
\] f�r $z�\in R$
\[\rightarrow \{x \in R|xa-ym=b\}=
	\begin{cases}
	R, & \text{falls }(a,m)=(0,0),\\
	rx'+zq &\text{falls }(a,m)\neq(0,0)
	\end{cases}
\] f�r $z \in R$
\[\implies \{x \in R|xa\equiv_{m}b\}=
	\begin{cases}
	R, & \text{falls }(a,m)=(0,0),\\
	rx'+Rq, & \text{falls }(a,m)\neq(0,0). \hfill{\Box}
	\end{cases}
\]
Damit ist die Aussage gezeigt.
\aufgabe{54}
\aufgabenteil{a} Bestimmen Sie einen Restklassenk�rper von $\mathbb{F}_{2}[X]$ mit 16 Elementen.
$K[X]/f=\{[r]|K[X]_{<~deg~f}\}$ ist eine Transversale gem�� 11.16.\\
Gesucht ist ein K�rper mit $|K|^{deg~f}$ Elementen, wobei $f$ irreduzibel ist, also keine Nullstellen besitzt.\\
$F_{2}[X]/_{X^4+X+1}=\{a+b[X]+c[X]^{2}+d[X]^{3}|a,b,c,d \in F_{2}\} \implies 2^{4} = 16$ Elemente.\\
$0^{4}+0+1=1, 1^{4}+1+1=3=1 \implies$ Keine Nullstelle in $F_{2}$. Also ist $F_{2}[X]/_{X^4+X+1}$ ein K�rper.
\aufgabenteil{b} Bestimmen Sie einen Restklassenk�rper von $\mathbb{F}_{4}[X]$ mit 16 Elementen.\\
$F_{4}[X]/X^2+1=\{a+b[X]|a,b \in \mathbb{F}_{4}\} \implies 4^{2} = 16$ Elemente.\\
$0^{2}+1=1, 1^{2}+1=2, 2^{2}+1=5=1, 3^{2}+1=10=2, 4^{2}+1=15=3 \implies$ Keine Nullstelle in $F_{4}[X]/X^2+1$, also liegt ein K�rper vor.



\end{document}
% Nummer des Blattes angepasst?