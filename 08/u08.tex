\documentclass[a4paper,graphics,11pt]{article}

\usepackage[latin1]{inputenc}
\usepackage[T1]{fontenc}
\usepackage{lmodern}
\usepackage[ngerman]{babel}
\usepackage{amsmath, tabu}
\usepackage{amsthm}
\usepackage{amssymb}
\usepackage{algorithm}
\usepackage{algpseudocode}
\usepackage{mathtools}
\usepackage{setspace}
\usepackage{graphicx,color,curves,epsf,float,rotating}
\usepackage{fullpage}
\floatname{algorithm}{Algorithmus}

\newcommand\norm[1]{\left\lVert#1\right\rVert}
\newcommand\abs[1]{\left\vert#1\right\vert}

\newcommand\aufgabe[1]{\subsection*{Aufgabe #1}}
\newcommand\aufgabenteil[1]{\subsubsection*{#1}}



\pagestyle{empty}
\begin{document}
\noindent WS 2016/17        							\hfill Simon Kaiser, 354692 \\
\noindent Gruppe \fbox{\textbf{3}}                      \hfill Johannes Salentin, 367543 \\

\begin{center}
\Large \textsc{Diskrete Strukturen} \\   % Fach
\large Aufgabenblatt 8                        % Nummer das Blattes, nicht vergessen zu �ndern!
\end{center}
\begin{center}
\rule[0.5ex]{\textwidth}{0.6pt}\vspace*{-\baselineskip}\vspace{3.2pt}
\rule[0.5ex]{\textwidth}{1.6pt}\\
\end{center}


%%%%%%%%%%%%%%%%%%%%%%%%%%%%%%%%%%%%%%
%
%   Ab hier kommt der Text
%   Neue Aufgabe mit \aufgabe{}
%   Aufgabenteil mit \aufgabenteil{}
% 
%%%%%%%%%%%%%%%%%%%%%%%%%%%%%%%%%%%%%%

\aufgabe{48}
Es sei $n \in \mathbb{N}_{0}$ und $\pi \in S_{n}$ mit $O:=[1,n] / \pi$ gegeben.
\aufgabenteil{a}
Jede Permutation ist eine Verkettung von Nachbartranspositionen
\begin{itemize}
\item Auf einer Bahn finden zyklische Vertauschungen statt. Also braucht man $m-1$ Transpositionen um eine Bahn der L�nge $m$ darzustellen. Insgesamt ben�tigt man somit f�r eine Permutation die Anzahl von Transpositionen von\\
$\displaystyle\sum_{o \in O} (|o|-1)$, wobei $|o|$ die L�nge einer Bahn ist.
\item Man kann Transpositionen weiter in Nachbartranspositionen zerlegen. Mit 12.20(b) ben�tigt man $2 \cdot |i-j|-1$ Nachbartranspositionen f�r eine Transposition.\\
Nach 12.25 ist $Inv((j,j+1))=\{(j,j+1)\}$ und damit gilt\\ $|Inv(i,j)|=2 \cdot |i-j|-1$ mit $(i,j) \in [1,n] \times [1,n]$.
\end{itemize}
Da eine Transposition in eine ungerade Anzahl von Nachbartranspositionen zerlegt werden kann und somit eine Transposition eine ungerade Anzahl von Fehlst�nden erzeugt, ist das Signum $-1$ f�r eine ungerade Anzahl von Transpositionen. Jede Permutation $\pi \in S_{n}$ l�sst sich nun durch $\sum_{o \in O} (|o|-1)$ Transpositionen darstellen ($1.$):
\begin{center}
$sgn(\pi)=(-1)^{\sum_{o \in O} (|o|-1)}$.
\end{center}
$|o|$ ist die L�nge einer Bahn. Ist diese Anzahl gleich 1, so kann diese Bahn einen Fehlstand haben und f�llt weg. Es gen�gt also die Bahnen mit L�nge $>1$ zu betrachten. 
\begin{center}
$(-1)^{\sum_{o \in O} (|o|-1)}=(-1)^{\sum_{o \in O,|o| > 1} (|o|-1)}$.
\end{center}
Au�erdem reicht es aus, die Bahnen mit gerader Elementenzahl zu betrachten, da diese durch eine ungerade Anzahl von Transpositionen dargestellt werden k�nnen ($1.$).\\
Da $sgn(\pi)=-1$ f�r ungerade viele Transpositionen, gilt
\begin{center}
$sgn(\pi)=(-1)^{|\{o \in O | |o| ist gerade\}|}$.
\end{center}
Auf einer einzelnen Bahn werden die Elemente zyklisch vertauscht. Das hei�t, eine Bahn der L�nge $m$ kann durch $m-1$ Transpositionen dargestellt werden. Wie bereits oben gezeigt, sind Transpositionen ungerade. Mit der Verkettungseigenschaft folgt:
\begin{center}
$sgn(\delta_{k})=(-1)^{m_{k}-1}$, wobei $ \pi=\displaystyle\prod_{k \in [1,|o|]}\delta_{k}$ und $ n=\displaystyle\sum_{k\in [1,|o|]} m_{k}$\\
$\implies sgn(\pi)=\displaystyle\prod_{k \in [1, |o|]} sgn(\delta_{k})=\displaystyle\prod_{k \in [1,|o|]} (-1)^{m_{k}-1}=(-1)^{\sum_{k�\in [1,|o|]}m_{k}-1}$\\
$= (-1)^{n-|o|}$.
\end{center}
Insgesamt gilt folglich:
\begin{center}
$sgn(\pi)=(-1)^{\sum_{o \in O}(|o|-1)}=(-1)^{\sum_{o \in O,|o| > 1}(|o|-1)}$\\
$=(-1)^{n-|o|}=(-1)^{|\{o \in O| |o|\text{ ist gerade}\}|}$.
\end{center}
\aufgabenteil{b}
\begin{center}
$(i) \iff \pi$ ist gerade $\iff|Inv(\pi)|$ ist gerade $\iff(-1)^{Inv(\pi)}=1=sgn(\pi)\iff (ii)$.
\end{center}
Wenn die Permutation $\pi$ gerade ist, muss auch die Anzahl der Fehlst�nde gerade sein. Das bedeutet wiederum, dass $sgn(\pi)=1$ ist. Also sind die Aussagen (i) und (ii) �quivalent.\\
Nach Teil $a$ gilt $sgn(\pi)=(-1)^{|\{o \in O| |o|\text{ ist gerade}\}|}$. Mit der Voraussetzung $(iii)$ gilt $sgn(\pi)=(-1)^{|\{o \in O | |o|\text{ ist gerade}\}|}=1$, was �quivalent zu $(ii)$ und damit zu $(i)$ ist.\\
Wenn die Permutation $\pi$ ein Kompositum einer geraden Anzahl von Transpositionen ist, so muss $\displaystyle\sum_{o \in o}(|o|-1)$ gerade sein (gem�� $a$).
\begin{center}
$sgn(\pi)=(-1)^{\sum_{o \in O}(|o|-1)}=1$ 
\end{center}
und somit �quivalent zu $(ii)$ und damit zu $(i)$ und $(iii)$.
Insgesamt sind folglich die Aussagen $(i)-(iv)$ �quivalent.
\aufgabe{49}
Es seien Gruppen $G$ und $H$ gegeben. Ein Gruppenhomomorphismus von $G$ nach $H$ ist eine Abbildung $ \varphi: G\to H$ derart, dass f�r $x,x' \in G$ stets
\begin{center}
$\varphi(x \cdot^{G}x')= \varphi(x) \cdot^{H} \varphi(x')$
\end{center}
gilt, kurz geschrieben als $\varphi(xx')=\varphi(x)\varphi(x')$.\\
Es seien abelsche Gruppen $A$ und $B$ gegeben. Ein Homomorphismus abelscher Gruppen von $A$ nach $B$ ist ein Gruppenhomomorphismus $\varphi: A \to B$. 
\aufgabenteil{a}
\begin{itemize}
\item F�r $n \in \mathbb{N}_{0}, \pi \in S_{n}$ ist $S_{n} \to S_{n}, \sigma \mapsto \pi \sigma$ ein Gruppenhomomorphismus.\\
\end{itemize}
Zu zeigen ist $\varphi(x\circ x')=\varphi(x) \circ \varphi(x')$.
\begin{center}
$\pi \circ x \circ x'=(\pi \circ x')$\\
$\iff_{\text{Assoziativgesetz}} \pi \circ x \circ x' = \pi \circ x \circ \pi \circ x'$. 
\end{center}
Seien
$\pi = 
	\begin{pmatrix}
	1&2&3 \\
	2&3&1
	\end{pmatrix}$
, $x =
	\begin{pmatrix}
	1&2&3 \\
	1&3&2
	\end{pmatrix}$
, $x' =
	\begin{pmatrix}
	1&2&3 \\
	3&2&1
	\end{pmatrix}$.
\begin{center}
$\pi \circ x \circ x' = 
	\begin{pmatrix}
	1&2&3 \\
	3&1&2
	\end{pmatrix}$
$\neq \begin{pmatrix}
	1&2&3 \\
	2&3&1
	\end{pmatrix}$
$= \pi \circ x \circ \pi \circ x'$.
\end{center}
Daraus folgt, dass kein Gruppenhomomorphismus vorliegt.
\begin{itemize}
\item F�r $n \in \mathbb{N}_{0}, \pi \in S_{n}$ ist $S_{n} \to S_{n}, \sigma \mapsto \pi \sigma \pi^{-1}$ ein Gruppenhomomorphismus.
\end{itemize}
Zu zeigen ist $\varphi(x \circ x') = \varphi(x) \circ \varphi(x')$.
\begin{center}
$\pi \circ (x \circ x') \circ \pi^{-1} = (\pi \circ x \circ \pi^{-1}) \circ \pi \circ x' \circ \pi^{-1}$\\
$\iff_{\text{ Ass.ges}} \pi \circ x \circ x' \circ \pi^{-1}= \pi \circ x \circ \pi^{-1} \circ \pi \circ x' \circ \pi^{-1}$\\
$\iff \pi \circ x \circ x' \circ \pi^{-1} = \pi \circ x \circ x' \circ \pi^{-1}$.
\end{center}
Daraus folgt, dass ein Gruppenhomomorphismus vorliegt.
\begin{itemize}
\item F�r $a \in \mathbb{Z}$ ist $\mathbb{Z} \to \mathbb{Z}, x \mapsto ax$ ein Homomorphismus abelscher Gruppen.
\end{itemize}
Zu zeigen ist $\varphi(x+x')=\varphi(x)+\varphi(x')$ gelten.
\begin{center}
$a+x+x'=a+x+a+x'$\\
$\iff_{\text{ K.ges}} a+x+x'\neq a+a+x+x'$. 
\end{center}
Daraus folgt, dass kein Homomorphismus abelscher Gruppen vorliegt.

\aufgabenteil{b}
Es sei ein Gruppenhomomorphismus $\varphi:G \to H$ gegeben.
\begin{itemize}
\item Es ist $\varphi(1)=1$.\\
$1 = \varphi(1) = \varphi(1 \cdot 1) = _{\text{ Gr.homo.}} \varphi(1) \cdot \varphi(1) = 1�\cdot 1 = 1$. $\Box$
\item F�r $x \in G$ ist $\varphi(x^{-1})=\varphi(x)^{-1}$.\\
$e_{H}= \varphi(e_{G})=\varphi(x \cdot x^{-1})=_{\text{ Gr.homo.}} \varphi(x) \cdot \varphi(x^{-1})$\\
$\implies \varphi(x^{-1})$ ist Inverses zu $\varphi(x) \implies \varphi(x^{-1})=\varphi(x)^{-1}$. $\Box$
\item Genau dann ist $\varphi$ injektiv, wenn f�r $x \in G$ aus $\varphi(x)=1$ bereits $x=1$ folgt.\\
Gem�� (i) gilt $\varphi(1)=1$. Injektivit�t ist erf�llt, wenn $\varphi(x)=\varphi(x') \implies x=x'$.	Kein anderes Element darf noch auf 1 abbilden.\\
$\varphi(x)=1 \implies x=1$. Also bildet nur ein Element auf 1 ab, da $\varphi(1)=1$ gilt.
\item Es ist $lm~\varphi$ eine Untergruppe von $H$.
$lm~\varphi=\{\varphi(x) \in H | x \in G\} \subseteq H$.\\
Die Gruppenaxiome Assoziativit�t, die Existenz eines neutralen Elements und die Existenz eines inversen Elements m�ssen erf�llt sein.\\
\emph{Assoziativit�t} gilt, da $Im~\varphi \subseteq H$.\\
\emph{Neutrales Element} ist $\varphi(1)=1$. Neutrales Element $\varphi(e_{G})=e_{H}$ immer im Bild enthalten.\\
\emph{Inverses Element} ist $\varphi(x^{-1})$ als Inverses zu $\varphi(x)$ (ii) f�r ein beliebiges $x \in G$.\\
Ergo ist $Im~\varphi$ eine Untergruppe von $H$.
\end{itemize}
\end{document}
% Nummer des Blattes angepasst?